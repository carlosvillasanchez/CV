%%%%%%%%%%%%%%%%%
% This is an sample CV template created using altacv.cls
% (v1.1.5, 1 December 2018) written by LianTze Lim (liantze@gmail.com). Now compiles with pdfLaTeX, XeLaTeX and LuaLaTeX.
%
%% It may be distributed and/or modified under the
%% conditions of the LaTeX Project Public License, either version 1.3
%% of this license or (at your option) any later version.
%% The latest version of this license is in
%%    http://www.latex-project.org/lppl.txt
%% and version 1.3 or later is part of all distributions of LaTeX
%% version 2003/12/01 or later.
%%%%%%%%%%%%%%%%

%% If you need to pass whatever options to xcolor
\PassOptionsToPackage{dvipsnames}{xcolor}

%% If you are using \orcid or academicons
%% icons, make sure you have the academicons
%% option here, and compile with XeLaTeX
%% or LuaLaTeX.
% \documentclass[10pt,a4paper,academicons]{altacv}

%% Use the "normalphoto" option if you want a normal photo instead of cropped to a circle
% \documentclass[10pt,a4paper,normalphoto]{altacv}

\documentclass[10pt,a4paper,ragged2e]{altacv}

%% AltaCV uses the fontawesome and academicon fonts
%% and packages.
%% See texdoc.net/pkg/fontawecome and http://texdoc.net/pkg/academicons for full list of symbols. You MUST compile with XeLaTeX or LuaLaTeX if you want to use academicons.

% Change the page layout if you need to
\geometry{left=1cm,right=9cm,marginparwidth=6.8cm,marginparsep=1.2cm,top=1.25cm,bottom=1.25cm}

% Change the font if you want to, depending on whether
% you're using pdflatex or xelatex/lualatex
\ifxetexorluatex
  % If using xelatex or lualatex:
  \setmainfont{Carlito}
\else
  % If using pdflatex:
  \usepackage[utf8]{inputenc}
  \usepackage[T1]{fontenc}
  \usepackage[default]{lato}
\fi

% Change the colours if you want to
\definecolor{NavyBlue}{HTML}{215290}
\definecolor{BodyBlue}{HTML}{4B7BB6}
\definecolor{SlateGrey}{HTML}{2E2E2E}
\definecolor{LightGrey}{HTML}{404040}
\colorlet{heading}{NavyBlue}
\colorlet{accent}{BodyBlue}
\colorlet{emphasis}{Black}
\colorlet{body}{LightGrey}

% Change the bullets for itemize and rating marker
% for \cvskill if you want to
\renewcommand{\itemmarker}{{\small\textbullet}}
\renewcommand{\ratingmarker}{\faCircle}

%% sample.bib contains your publications
\addbibresource{sample.bib}

\begin{document}

\name{Carlos Villa Sánchez}
%\tagline{Mr Algorithms}
%\photo{2.8cm}{Globe_High}
\personalinfo{%
  % Not all of these are required!
  % You can add your own with \printinfo{symbol}{detail}
  \email{carlos.villa.vs@gmail.com}
  \phone{+34 676985244}
  %\mailaddress{Address, Street, 00000 County}
  \location{Madrid, Spain}
  %\homepage{www.homepage.com/}
  %\twitter{@twitterhandle}
  \linkedin{linkedin.com/in/carlos-villa-sánchez}
  \github{github.com/carlosvillasanchez}
  %% You MUST add the academicons option to \documentclass, then compile with LuaLaTeX or XeLaTeX, if you want to use \orcid or other academicons commands.
  % \orcid{orcid.org/0000-0000-0000-0000}
}

%% Make the header extend all the way to the right, if you want.
\begin{fullwidth}
\makecvheader
\end{fullwidth}

%% Depending on your tastes, you may want to make fonts of itemize environments slightly smaller
% \AtBeginEnvironment{itemize}{\small}

%% Provide the file name containing the sidebar contents as an optional parameter to \cvsection.
%% You can always just use \marginpar{...} if you do
%% not need to align the top of the contents to any
%% \cvsection title in the "main" bar.
\cvsection[page1sidebar]{Experience}

\cvevent{Main Programmer and Co-owner}{Presup project}{February 2019 -- April 2020}{Madrid}

{In this project, we have developed a desktop application for the main operating systems that helps translators in the generation of budgets, analysing documents prior to translation. Project developed in Python.} 

\divider

\cvevent{Assistant Researcher Internship in Digitization Processes}{Universidad Politécnica de Madrid}{September 2018 -- July 2019}{ETSIT, Madrid}

{Member of the "Digital Integration Group" (GID) of the ETSIT, UPM. In this team, we aimed to carry out a digitization of the university, centralizing the various applications used by students, teachers and staff in three portals for each type of member and, also, developing and maintain new web applications.} 


\divider

\cvevent{External Scholarship in Cybersecurity}{Accenture Security}{February 2018 -- July 2018}{La Finca Bussiness Park, Madrid}

{Member of the security team inside a project for an e-commerce.}

\medskip
\cvsection{Education}

\cvevent{Master in Telecomunication Enigineering}{École Polytechnique Fédérale de Lausanne, Universidad Politécnica de Madrid}{September 2018 -- Ongoing}{EPFL, Lausanne; ETSIT, Madrid}
Two years master. First one in the UPM, Madrid; second one in EPFL, Lausanne. 


\divider

\cvevent{Bachelor of Engineering in Telecommunication Technologies and Services}{Universidad Politécnica de Madrid}{September 2014 -- June 2018}{ETSIT, Madrid}
Four years bachelor.





\medskip
\cvsection{Technologies}

\cvtag{Python}
\cvtag{JavaScript}
\cvtag{Linux}
\cvtag{Go}
\cvtag{Java}
\cvtag{C}
\cvtag{Networking}
\cvtag{Computer and Network Security}
\cvtag{Single Sing-on}
\cvtag{TCP/IP}
\cvtag{Docker}
\cvtag{CAS Protocol}
\cvtag{Blockchain}
\cvtag{SQL}
\cvtag{NoSQL}
\cvtag{Machine Learning}
\cvtag{Advanced cryptography}
\cvtag{Bash}
\cvtag{Scrum}
\cvtag{Git}
\cvtag{Software security}

\medskip
\cvsection{Certifications}
{TOEFL iBT: Score 111/120 (Dic 2018)}

%% If the NEXT page doesn't start with a \cvsection but you'd
%% still like to add a sidebar, then use this command on THIS
%% page to add it. The optional argument lets you pull up the
%% sidebar a bit so that it looks aligned with the top of the
%% main column.
% \addnextpagesidebar[-1ex]{page3sidebar}


\end{document}
